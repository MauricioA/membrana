\chapter{Modelo Acoplado}
% CAP 6 Todo acoplado. Resultados. poner snapshots de valores 1-9

En este capítulo se realizan simulaciones de todos los fenómenos físicos simulados en los capítulos anteriores juntos, es decir el potencial eléctrico, la creación de poros en la membrana y la posterior variación en el radio de los mimos, las variaciones en la conductividad y difusión de la membrana producto de la aparición de poros y el transporte de especies iónicas en el dominio. Además se estudió el efecto de aplicar varios pulsos consecutivos a través de los electrodos, en lugar de un único pulso.

\section{Implementación}
Para realizar la simulación completa se creó un ciclo principal que realiza llamados a las diferentes rutinas implementadas en los capítulos anteriores.\\

Para acelerar los tiempos de ejecución se usaron intervalos temporales diferentes para los distintos fenómenos físicos simulados. Para el cálculo de la densidad de poros en la membrana celular y sus radios se utilizó un intervalo temporal muy pequeño, ya que se notó que al aumentarlo se producen errores de discretización muy grandes que producen la divergencia del sistema. Para el cálculo de los potenciales eléctricos en el dominio, en cambio, alcanzó un intervalo temporal mucho mayor, y se notó que reducirlo no impacta en los resultados. Por último para el cálculo de las concentraciones de las especies iónicas se necesitaron intervalos temporales aún mayores. Concretamente se usaron intervalos de 1 \si{\nano\second} para las ecuaciones de poros, 20 \si{\nano\second} para el transporte de especies y un intervalo variable de entre 1 \si{\nano\second} y 2 \si{\micro\second} para el cálculo del potencial eléctrico. Este último intervalo es variable dado que los primeros instantes de cada pulso eléctrico es el que tiene mayores variaciones en los potenciales producto de la repentina permeabilización de la membrana, como puede verse en el capítulo 4. Esto hace que se necesite una actualización constante de los valores de potencial en los nodos, sobre todos los cercanos a la membrana celular. Sin embargo pasados los primeros microsegundos del pulso el sistema se estabiliza y la permeabilización y valores de PTM se mantienen con pocos cambios lo que hace innecesario un cálculo constante de los potenciales. Por esta razón se optó por un intervalo muy pequeño al comenzar un pulso y se incrementa de manera exponencial.\\

%TODO revisar los valores de los intervalos!! revisar si transporte efectivamente mayor que poisson
%TODO está bien decir delta t exponencial? aumenta como un capacitor

%TODO on time, off time. mencionar que se reinicia el delta_t

%TODO mencionar truco numérico temperatura

\section{Resultados}

Se presentan a continuación las concentraciones en el dominio de las diferentes especies iónicas estudiadas para diferentes instantes de tiempo. Los resultados corresponden a simulaciones de una célula de ??? bajo ? pulsos eléctricos con una diferencia potencial de ??? y una duración de ??? encendidos y ??? apagados cada uno.\\

Tener primero resultados para poder escribir.

\subsection*{\h}

Acá poner snapshots de pH \\

\subsection*{\oh}

\subsection*{\na}

\subsection*{\cl}
