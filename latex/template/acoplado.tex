\chapter{Modelo Acoplado}
% CAP 6 Todo acoplado. Resultados. poner snapshots de valores 1-9

En este capítulo se realizan simulaciones de todos los fenómenos físicos simulados en los capítulos anteriores juntos, es decir el potencial eléctrico, la creación de poros en la membrana y la posterior variación en el radio de los mismos, las variaciones en la conductividad y difusión de la membrana producto de la aparición de poros y el transporte de especies iónicas en el dominio. Además se estudió el efecto de aplicar varios pulsos consecutivos a través de los electrodos, en lugar de un único pulso.

\section{Implementación}
Para realizar la simulación completa se creó un ciclo principal que realiza llamados a las diferentes rutinas implementadas en los capítulos anteriores.\\

Para acelerar los tiempos de ejecución se usaron intervalos temporales diferentes para los distintos fenómenos físicos simulados. Para el cálculo de la densidad de poros en la membrana celular y sus radios se utilizó un intervalo temporal muy pequeño, ya que se notó que al aumentarlo se producen errores de discretización muy grandes que producen la divergencia del sistema. Para el cálculo de los potenciales eléctricos en el dominio, en cambio, alcanzó un intervalo temporal mucho mayor, y se notó que reducirlo no impacta en los resultados. Por último para el cálculo de las concentraciones de las especies iónicas se necesitaron intervalos temporales aún mayores. Concretamente se usaron intervalos de 1 \si{\nano\second} para las ecuaciones de poros, 20 \si{\nano\second} para el transporte de especies y un intervalo variable de entre 1 \si{\nano\second} y 2 \si{\micro\second} para el cálculo del potencial eléctrico. Este último intervalo es variable dado que los primeros instantes de cada pulso eléctrico es el que tiene mayores variaciones en los potenciales producto de la repentina permeabilización de la membrana, como puede verse en el capítulo 4. Esto hace que se necesite una actualización constante de los valores de potencial en los nodos, sobre todos los cercanos a la membrana celular. Sin embargo pasados los primeros microsegundos del pulso el sistema se estabiliza y la permeabilización y valores de PTM se mantienen con pocos cambios lo que hace innecesario un cálculo constante de los potenciales. Por esta razón se optó por un intervalo muy pequeño al comenzar un pulso y se incrementa de manera exponencial.\\

%TODO revisar los valores de los intervalos!! revisar si transporte efectivamente mayor que poisson
%TODO está bien decir delta t exponencial? aumenta como un capacitor

%TODO on time, off time. mencionar que se reinicia el delta_t

%TODO mencionar truco numérico temperatura

\section{Resultados}

\newcommand{\lineasnap}[7]{
	#2 &
	\includegraphics[width=0.19\textwidth]{#1#3} & 
	\includegraphics[width=0.19\textwidth]{#1#4} & 
	\includegraphics[width=0.19\textwidth]{#1#5} & 
	\includegraphics[width=0.19\textwidth]{#1#6} & 
	\includegraphics[width=0.19\textwidth]{#1#7} \\
}

\begin{table} \begin{center} 
	\begin{tabular}
		{ m{0.5cm} >{\centering\arraybackslash}m{0.17\textwidth} >{\centering\arraybackslash}m{0.17\textwidth} >{\centering\arraybackslash}m{0.17\textwidth} >{\centering\arraybackslash}m{0.17\textwidth} >{\centering\arraybackslash}m{0.17\textwidth} }
		& 1\ms & 2\ms & 3\ms & 4\ms & 5\ms \\
		\lineasnap{acoplado/1p160kvm/h} {\h} {10}{20}{30}{40}{50}
		\lineasnap{acoplado/1p160kvm/oh}{\oh}{10}{20}{30}{40}{50}
		\lineasnap{acoplado/1p160kvm/na}{\na}{10}{20}{30}{40}{50}
		\lineasnap{acoplado/1p160kvm/cl}{\cl}{10}{20}{30}{40}{50}
	\end{tabular}
	\caption{Concentraciones en diferentes instantes de tiempo}
	\label{tbl:snap1}
\end{center} \end{table}

Se presentan a continuación las concentraciones en el dominio de las diferentes especies iónicas estudiadas para diferentes instantes de tiempo. Los resultados corresponden a simulaciones de una célula de 25\usec de radio bajo un pulsos eléctrico que genera un campo de 1600\vcm y una duración de 5\ms. Las imágenes de la tabla \ref{tbl:snap1} corresponden a las concentraciones molares de las cuatro especies en diferentes instantes del pulso. Los colores azules corresponden a los extremos bajos de concentración y los rojos a los extremos altos.

Se observan en todos los casos cambios en el interior de la célula mucho mayores a los obtenidos en las simulaciones del capítulo \ref{chap:trans}. Esto se debe a que la simulación de la generación de poros posibilitó la permeabilización de la membrana y el ingreso o egreso con mayor facilidad de las especies iónicas. 

En el caso del \h se observa un gran movimiento en las concentraciones, obteniéndose valores muy altos de concentración en la región hiperpolarizada del interior de la célula, que se observan con una franja roja cercana a la membrana. En cuanto a la región depolarizada del interior, la concentración de \h disminuyó notablemente, lo que se observa con una mancha de color verde que avanza hacia el ecuador de la célula con el paso del tiempo. El líquido extracelular también tuvo grandes cambios, con concentraciones altas en la región depolarizada y bajas en la región hiperpolarizada (al contrario del interior).

Las concentraciones de \oh presentan un comportamiento opuesto al observado en el \h, es decir en el interior de la célula se producen extremos altos de concentración en el polo depolarizado y una disminución de materia en el polo hiperpolarizado, mientras que en el exterior se alcanzan extremos altos cerca de la región de la membrana cercana al polo positivo y extremos bajos en el polo negativo. 

En cuanto a las concentraciones de \na y \cl, se observa mucho menos movimiento, dado que sus constantes de difusión son mucho menores que las del \h y \oh. Al igual que en los casos anteriores se tienen concentraciones extremas cerca de la membrana, con el mismo patrón de comportamiento según el signo de la especia iónica (el \na se comporta como el \h por ser de carga positiva y el \cl como el \oh por ser de carga negativa). No se observa, sin embargo, una zona de concentraciones bajas que avance hacia el centro de la célula, como en los casos anteriores, pero se alcanza a notar que las zonas con valores extremos son de mayor tamaño que las obtenidas en el capítulo \ref{chap:trans}, en el que no se tenía en cuenta la permeabilización de la membrana.

Si bien se observan cambios significativos en las concentraciones de las especies en el interior de la célula, estos cambios se dan principalmente en las regiones cercanas a la membrana. Sin embargo se observa que con el paso del tiempo las regiones con valores extremos se vuelven mayores, lo cuál indica que se podrían obtener mayores cambios en las concentraciones internas con pulsos de mayor duración o con varios pulsos. 

%TODO subir a share (tesis, transporte y acoplado) y github

%TODO VARIOS PULSOS!!!!