\appendix
\chapter{Instrucciones de uso} \label{chap:uso}

El código entregado se puede compilar en Visual Studio importando la solución con el archivo \texttt{celula.sln}, que usa el compilador Visual C++. Se puede compilar y correr en Linux con el compilador GCC usando el archivo \texttt{Makefile}. Para compilar con Intel C Compiler es necesario modificar el \texttt{Makefile} o el proyecto de Visual Studio. Adicionalmente también se incluyen varios scripts en lenguaje Python para ayudar al proceso de generar de mallas, procesar las salidas obtenidas y generar gráficos.

\section*{Parámetros de entrada}

El programa recibe los parámetros de entrada a través de un archivo \texttt{input.in}, que consta de varias líneas con información de la corrida en la forma \texttt{parámetro: valor} donde los parámetros son los siguientes:

\begin{description}
	\item[\texttt{malla}] ruta del archivo con la malla que representa el dominio. 
	\item[\texttt{nodpel}] cantidad de nodos por elemento. La mayor parte del código solo puede correr con elementos de 4 nodos, pero el subproblema del potencial eléctrico también puede ser ejecutado con elementos de 3 nodos.
	\item[\texttt{problema}] tipo de problema a correr. Puede ser \texttt{potencial} si solo se desean obtener los resultados de potencial eléctrico (capítulo \ref{chap:itv}), \texttt{poros} si se desea resolver el potencial y el crecimiento de poros (capítulo \ref{chap:poros}), \texttt{transporte} si se desea resolver el potencial y transporte de especies sin poros (capítulo \ref{chap:trans}) y \texttt{acoplado} si se desean correr el modelo acoplado con todos los subproblemas (capítulo \ref{chap:acoplado}).
	\item[\texttt{salida}] ruta al directorio de salida. El programa no ejecuta si el directorio de salida no está vacío (para evitar sobrescribir resultados).
	\item[\texttt{threads}] cantidad de hilos de procesamiento a utilizar. Se recomienda usar tantos como tenga el procesador.
	\item[\texttt{delta\_t}] intervalo temporal usado para el cálculo de densidad y radio de poros. Las demás ecuaciones se resuelven con intervalos temporales mayores, que son múltiplos del intervalo de poros. Si sólo se calcula el potencial eléctrico este valor se ignora. Para los resultados de este trabajo se utilizó un valor de 1 \si{\nano\second}.
	\item[\texttt{sigint}] valor de conductividad eléctrica en el interior de la célula, expresado en \si{\siemens \per \micro\metre}.
	\item[\texttt{sigext}] valor de conductividad eléctrica del líquido extracelular, expresado en \si{\siemens \per \micro\metre}.
	\item[\texttt{sigint}] valor de conductividad eléctrica en el la membrana celular, expresado en \si{\siemens \per \micro\metre}.
	\item[\texttt{potencial}] diferencia de potencial entre los electrodos, medida en \si{\volt}. Se asume que se encuentra el electrodo positivo en el borde superior de la malla y el negativo en el borde inferior.
	\item[\texttt{radio}] radio de la célula, medido en \si{\micro\metre}.
	\item[\texttt{ancho}] ancho de la membrana celular, medido en \si{\micro\metre}.
	\item[\texttt{pulsos}] cantidad de pulsos a simular.
	\item[\texttt{on\_time}] tiempo que estará prendido cada uno de los pulsos, medido en segundos.
	\item[\texttt{off\_time}] tiempo que estará apagado cada uno de los pulsos, medido en segundos. 
\end{description}

Los parámetros pueden estar en cualquier orden y puede haber en el archivo \texttt{input.in} comentarios que comiencen con \texttt{\#}. La malla del dominio debe ser un archivo de texto con el siguiente formato:

\begin{itemize}
	\item Una línea con la cantidad de nodos.
	\item Una línea con cada nodo indicando el número de nodo y las posiciones en $x$ e $y$ separados por espacios.
	\item Una línea con la cantidad de zonas del dominio.
	\item Una línea por cada zona del dominio, con un número de zona y la cantidad de elementos de la zona, separados por espacios.
	\item Una línea por cada elemento de cada zona, con un número de elemento dentro de la zona y los números de los nodos que componen el elemento, separados por espacios.
\end{itemize}

\section*{Formatos de salida}

El programa genera varias carpetas y archivos en el directorio de salida indicado:

\begin{itemize}
	\item En el directorio \texttt{tension} genera cada 100\usec de la simulación archivos con el nombre \texttt{tension.csv.xxx} con \texttt{xxx} números consecutivos. Los archivos tienen valores de potencial eléctrico para cada nodo, indicado con coordenadas $x$ e $y$.
	\item En el directorio \texttt{campo} genera cada 100\usec archivos con nombre \texttt{campo.csv.xxx} con valores de las componentes horizontal y vertical y módulo del campo eléctrico y valores de corriente de cada elemento, indicado con coordenadas $x$ e $y$.
	\item En el directorio \texttt{poros} genera archivos con nombre \texttt{posos.csv.xxx} cada 100\usec con el ángulo polar y el radio en \um de cada poro de la membrana. 
	\item En el directorio \texttt{concentracion} genera archivos \texttt{concentracion.csv.xxx} cada 100\usec con los valores de concentración de \h, \oh, \na y \cl en cada nodo expresados en átomos por \si{\micro\metre\cubed}.
	\item En el directorio \texttt{concentracion} genera archivos \texttt{molar.csv.xxx} con los mismos valores de concentración pero expresados como concentraciones molares.
	\item En el directorio \texttt{concentracion} genera archivos con los valores de pH y pOH de cada nodo.
	\item En un único archivo \texttt{itv.csv} graba periódicamente valores de potencial transmembrana indicados con columnas de tiempo, ángulo polar y PTM.
	\item Genera una copia del archivo \texttt{input.in} utilizado.
\end{itemize}

%esto mejor no poner: por salida se imprimen periodicamente el tiempo de la simulación, el error en el cálculo de transporte, la cantidad de poros, y el tiempo por iter en promedio
