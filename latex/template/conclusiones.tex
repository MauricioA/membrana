\chapter{Conclusiones}

En su estado actual, el código desarrollado permite resolver sobre un dominio bidimensional que contempla simetría de revolución el potencial sobre la membrana celular y el líquido intra y extracelular. Así mismo provee la respuesta dinámica de la creación y evolución de la población de poros de la membrana a dicho potencial aplicado. Por último analiza el transporte de cuatro especie iónicas a través de la membrana.

El modelo discretiza la membrana explícitamente, utilizando un mallado adaptativo. Debemos destacar que este modo de simular el problema es inédito ya que hasta el momento es usual en la literatura utilizar dos medios (que corresponden al líquido intra y extracelular) considerar a la membrana como una condición de contorno sobre la que se supone un potencial PTM. Nuestro modelo, a partir de la distribución de potencial y campo eléctrico sobre el dominio, reproduce adecuadamente tanto la forma como la magnitud del PTM. Además de predecir el valor umbral, mas allá del cual no varía apreciablemente a pesar de aumentar la diferencia de potencial en los electrodos. \todo[inline]{cuidado que lo del umbral no está hecho}

El mallado de elementos finitos seleccionado para realizar los cálculos es óptimo, pues combina un grado de precisión adecuado con estabilidad numérica. Siempre es posible extender el numero de elementos y nodos con lo cual se aumentaría el tamaño de los sistema lineales a resolver, y con ello el tiempo de cómputo. Sin embargo no obtendríamos con ello en nuestra configuración de parámetros un aumento en la calidad de los resultados.

En cuanto al modelado de la dinámica y creación de la población de poros, se comprobó que la evolución de la densidad de poros es la adecuada para explicar la existencia de este umbral y que el potencial aplicado influye en la velocidad de la evolución tanto de la densidad como del radio de los poros. 
Es necesario observar que la mayoría de los poros creados se sellan muy rápidamente por lo que no influyen en el ingreso de iones dentro de la célula, por lo que si se pretende permeabilizar la membrana para las especies estudiadas, se vuelve esencial repetir el pulso periódicamente. 

El transporte de especies a través de la membrana dado por mecanismos guiados por la difusión y la movilidad de las especies iónicas responde adecuadamente a la aplicación de pulsos equi-espaciados temporalmente, observándose que la reapertura de los poros permite el reingreso de material dentro de la célula. Estos resultados responden cualitativamente a los medidos experimentalmente.\\ 

Hay muchos tipos de análisis que quedaron pendientes y que podrían ser realizados en trabajos futuros. Entre ellos se destaca una mayor parametrización de las múltiples variables involucradas; sobre todo los pulsos eléctricos. Es posible analizar diferentes tipos de pulsos, de mayor o menor frecuencia, con potenciales más variados, con diferentes relaciones entre los tiempos de apagado y prendido y diferentes tipos de ondas. También se podría analizar de qué manera una célula puede bloquear el ingreso de especies a otra célula, o reducir el efecto de electroporación. Para esto sería necesario simular un dominio con más de una célula, lo cuál se puede realizar con facilidad con el código implementado. 

Otros tipos de análisis pendientes requerirían en cambio un mayor trabajo. Uno de estos tipos de análisis es el modelado de células con formas irregulares, como es comúnmente el caso en los tejidos. Esto obligaría a trabajar con un sistema de coordenadas tridimensional, dado que las células irregulares no son sólidos de revolución y no se podría aprovechar el sistema de coordenadas cilíndricas. De esta manera sería necesario reescribir gran parte del código implementado y optimizar los métodos numéricos, ya que la cantidad de elementos en la malla se incrementaría al trabajar en tres dimensiones. Por último se podría analizar también la deformación celular producida por la diferencia de potencial y se podrían modelar las organelas internas de la célula para obtener un modelo más realista.
