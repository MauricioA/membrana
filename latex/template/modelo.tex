\chapter{Descripción del Modelo} \label{chap:modelo}
% CAP 2 descripción del modelo. qué se resuelve, método numérico. Explicación de FEM
% podría ir teoría de gradientes conjugados, etc

% section con explicación de todo el trabajo (solo teoría, nada de imple). debería ir con las fórmulas y todo. podría ir dibujito de célula con ángulo \theta
%\subsection{Potencial eléctrico}

% El chapter anterior debería tener una explicación a grandes rasgos de que es lo que se hace en el trabajo

\section{Modelo Matemático}

\subsection*{Potencial Eléctrico}
El campo eléctrico en el dominio es generado por dos electrodos con un potencial constante durante la duración del pulso. El potencial eléctrico en todo el dominio se calcula según la ecuación 

\begin{equation} \label{eq:poisson}
	\nabla \sigma_{elem} \cdot (\nabla \phi) = 0 
\end{equation}

donde $\phi$ representa el potencial eléctrico y $\sigma_{elem}$ la conductividad del material, para $elem = o, i$ o $m$ para el líquido extracelular, el citoplasma o la membrana celular respectivamente.\\

La diferencia de potencial entre el interior y el exterior de la célula en un punto de su superficie se conoce como potencial transmembrana (PTM). Si la célula es esférica este potencial se puede aproximar con la fórmula cerrada

\begin{equation} \label{eq:cos}
	V^{\theta} = 1.5\, E\, \alpha\, cos (\theta)
\end{equation}

% puede ir \cite{puchiar}
donde $E$ es el campo eléctrico, $\theta$ el ángulo polar respecto del campo eléctrico y $\alpha$ el radio de la célula. Esta fórmula no tiene en cuenta que el PTM puede variar en el tiempo por la creación de poros, por eso en este trabajo no se la usa directamente, si no que usa la ecuación \ref{eq:poisson}.

\subsection*{Generación y evolución de poros}
Si el PTM es lo suficientemente alto, genera poros hidrofílicos en la membrana celular, que se pueden calcular según la ecuación

\begin{equation} \label{eq:poros-crea}
	\frac{\partial N}{\partial t} = \alpha_c e^{(V_m/V_{ep})^2} \left( 1 - \frac{N}{N_0 e^{q \left(V_m/V_{ep} \right) ^2}} \right)
\end{equation}

%poner cita \cite{krass}
donde $N$ es la densidad de poros en un determinado tiempo y posición de la membrana celular, $\alpha_c$ es el coeficiente de creación de poros, $V_m$ es el potencial transmembrana, $V_{ep}$ es el voltaje característico de electroporación, $N_0$ es la densidad de poros en equilibrio (cuando $V_m = 0$) y $q$ es una constante igual a $(r_m / r*)^2$, donde $r_m$ es el radio de mínima energía para $V_m = 0$ y $r*$ es el radio mínimo de los poros.\\

Los poros se crean con un radio inicial $r*$ y su radio varía en el tiempo según el potencial transmembrana de acuerdo a la ecuación

\begin{equation} \label{eq:poros-radio}
	\frac{\partial r}{\partial t} = \frac{D}{kT} \left( \frac{V_m^2 F_{max}}{1+r_h / (r+r_a)} + \frac{4 \beta}{r} \left(\frac{r_*}{r}\right)^4 - 2 \pi \gamma + 2 \pi \sigma_{\textrm{\tiny eff}} r\right)
\end{equation}

donde $r$ es el radio de un poro, $D$ es el coeficiente de difusión para los poros, $k$ es la constante de Boltzmann, $T$ la temperatura absoluta, $V_m$ el potencial transmembrana, $F_{max}$ la máxima fuerza eléctrica para $V_m$ de 1V, $r_h$ y $r_a$ son constantes usadas para la velocidad de advección, $\beta$ es la energía de repulsión estérica, $\gamma$ es la energía del perímetro de los poros, y $\sigma_{\textrm{\tiny eff}}$ es la tensión efectiva de la membrana, calculada como

\begin{equation}
	\sigma_{\textrm{\tiny eff}} = 2 \sigma^\prime - \frac{2 \sigma^\prime - \sigma_0}{(1 - A_p / A)^2}
\end{equation}

% puede ir \cite{krass}
donde $\sigma^\prime$ es la tensión de la interfase hidrocarburo-agua, $\sigma_0$ es la tensión de la bicapa sin poros, $A_p$ es la suma de las áreas de todos los poros en la célula, y $A$ es el área de la célula. En la ecuación \ref{eq:poros-radio}, el primer término corresponde a la fuerza eléctrica inducida por el potencial transmembrana, el segundo a la repulsión estérica, el tercero a la tensión de línea que actúa en el perímetro del poro y el cuarto a la tensión superficial de la célula.\\

Por otra parte se asume que la membrana celular se carga como un capacitor y una resistencia en paralelo. De esta manera el potencial transmembrana no aumenta bruscamente al iniciarse el pulso eléctrico, si no que crece de manera paulatina según la ecuación: 

\begin{equation} \label{eq:capacit} \begin{split}
	V_m = V_p\, (1 - e^{-t/\tau}) , \\ \textrm{con } \tau = \alpha\, C_m \left( \frac{1}{\sigma_i} + \frac{1}{2 \sigma_o} \right)
\end{split} \end{equation}

%poner \cite{krass}
donde $V_m$ es el potencial transmembrana en un punto de la superficie de la célula, $V_p$ es el potencial obtenido por las ecuaciones de potencial eléctrico en ése mismo punto, $t$ es el tiempo transcurrido desde el comienzo del pulso eléctrico, $\alpha$ es el radio de la célula, $C_m$ es la capacitancia superficial de la célula y $\sigma_i$ y $\sigma_o$ las conductancias intra y extracelulares respectivamente.\\

\subsection*{Transporte de especies}
Para conocer la concentración de las especies iónicas se usa la ecuación de conservación de masa de Nernst-Planck:

\begin{equation} \label{eq:trans}
	\frac{\partial C_i}{\partial t} = \nabla \cdot \left( D_i \nabla C_i + D_i z_i \frac{F}{R T} C_i \nabla \phi \right)
\end{equation}

%\cite{fodava} abajo
donde $C_i$, $D_i$ y $z_i$ representan la concentración, el coeficiente de difusión y la valencia respectivamente de la especie $i$, para $i = $ \h, \oh, \na ó \cl.
$F$ es la constante de Faraday, $R$ la constante de los gases y $T$ la temperatura. 
Esta ecuación tiene en cuenta la difusión de las partículas (con el término $D_i \nabla C_i$) pero también el efecto de migración producto del campo eléctrico (con el término $D_i z_i \frac{F}{R T} C_i \nabla \phi$).\\

\subsection*{Condiciones de borde}
%TODO ESTA MAL LA REFERENCIA A LA EQ POISSON!!
Para la ecuación \ref{eq:poisson} se usan condiciones de borde de Dirichlet con potenciales fijos en los electrodos, mientras que para el borde no ocupado por electrodos se usan condiciones de borde de Neumann:

\begin{equation}
	\frac{\partial \phi}{\partial \mathbf{n}} = 0
\end{equation}

donde $\mathbf{n}$ representa la normal al borde.\\

Para la ecuación de generación de poros \ref{eq:poros-crea} se usa como condición inicial que la membrana no contiene poros, mientras que para la ecuación \ref{eq:poros-radio} se asume que los poros se crean con un radio inicial $r^*$.\\

Para la ecuación de transporte de especies \ref{eq:trans} se usan como condiciones iniciales las concentraciones descritas en la tabla \ref{table:tablita}: $C_{e, i}^0$ siendo $e =$ $i$ ó $o$ si se refiere a los nodos del interior o del exterior de la célula respectivamente e $i =$ \h, \oh, \na ó \cl para la concentración de cada especie y $C_{e,i}$ con $e =$ $a$ ó $c$ si es para el ánodo o el cátodo respectivamente.\\

Como condición de borde en el borde no ocupado por los electrodos se usa
\begin{equation}
	\frac{\partial C_i}{\partial \mathbf{n}} = 0
\end{equation}

Para los bordes ocupados por los electrodos se usan los valores fijos $C_{e,i}$ descritos anteriormente.


\subsection{Constantes}
A continuación se presenta la definición y valores de las constantes usadas.

TODO CORREGIR LOS 0 1 QUE ESTÁN MAL! (ver preinforme)

TODO revisar todos los valores de la tabla!

\newcommand{\lineaTabla}[3]{ ${#1}$ & {#3} & {#2} \\ }

\newcommand{\anodo}[3] {
	\lineaTabla{C_{a,{#1}}}{\num{#2} \si{#3}}{Concentración de #1 en el ánodo}
}

\newcommand{\catodo}[3] {
	\lineaTabla{C_{c,{#1}}}{\num{#2} \si{#3}}{Concentración de #1 en el cátodo}
}

\begin{table}
    \centering
	\begin{tabular}{|l l l|} 
		\hline Símbolo & Definición & Valor \\
		\hline
				
		\lineaTabla{\sigma_{o}}{0.20 \si{\siemens\per\metre}}{Conductividad de la zona extracelular}
		\lineaTabla{\sigma_{i}}{0.15 \si{\siemens\per\metre}}{Conductividad de la zona intracelular}
		\lineaTabla{\sigma_{m}}{\num{5e-7} \si{\siemens\per\metre}}{Conductividad de la membrana celular}
		\lineaTabla{\sigma_{p}}{2 \si{\siemens\per\metre}}{Conductividad del líquido que llena el poro}
		\lineaTabla{E}{40 \si{\kilo\volt\per\metre} - 200 \si{\kilo\volt\per\metre}}{Campo eléctrico aplicado}
		\lineaTabla{\alpha}{10 \si{\micro\metre} - 50 \si{\micro\metre}}{Radio de la célula}
		\lineaTabla{d}{5 \si{\nano\metre}}{Ancho de la membrana}
		
		\lineaTabla{D_\h}{\num{12500} \si{\micro\metre\per\metre^{2}}}{Coeficiente de difusión para \h}
		\lineaTabla{D_\oh}{\num{7050} \si{\micro\metre\per\metre^{2}}}{Coeficiente de difusión para \oh}
		\lineaTabla{D_\na}{\num{1780} \si{\micro\metre\per\metre^{2}}}{Coeficiente de difusión para \na}
		\lineaTabla{D_\cl}{\num{3830} \si{\micro\metre\per\metre^{2}}}{Coeficiente de difusión para \cl}		

		\lineaTabla{C_{i,\h}^0}{\num{.3978e-7} \si{\textsc{m}}}{Concentración inicial de \h en citoplasma}
		\lineaTabla{C_{i,\oh}^0}{\num{.3978e-7} \si{\textsc{m}}}{Concentración inicial de \oh en citoplasma}
		\lineaTabla{C_{i,\na}^0}{\num{142} \si{\milli\textsc{m}}}{Concentración inicial de \na en citoplasma}
		\lineaTabla{C_{i,\cl}^0}{\num{108} \si{\milli\textsc{m}}}{Concentración inicial de \cl en citoplasma}

		\lineaTabla{C_{o,\h}^0}{\num{1e-7} \si{\textsc{m}}}{Concentración inicial externa de \h}
		\lineaTabla{C_{o,\oh}^0}{\num{1e-7} \si{\textsc{m}}}{Concentración inicial externa de \oh}
		\lineaTabla{C_{o,\na}^0}{\num{14e-7} \si{\milli\textsc{m}}}{Concentración inicial externa de \na}
		\lineaTabla{C_{o,\cl}^0}{\num{4e-7} \si{\milli\textsc{m}}}{Concentración inicial externa de \cl}
	
		\anodo{\h}{1.5e7}{at.\micro\metre^{-3}}
		\anodo{\oh}{0}{}
		\anodo{\na}{1e12}{at.\micro\metre^{-3}}
		\anodo{\cl}{0}{}

		\catodo{\h}{0}{}
		\catodo{\oh}{1.806e7}{at.\micro\metre^{-3}}
		\catodo{\na}{0}{}
		\catodo{\cl}{0}{}
		
		\lineaTabla{r*}{0.51 \si{\nano\metre}}{Radio mínimo de los poros}
		\lineaTabla{r_m}{0.80 \si{\nano\metre}}{Radio del poro de mínima energía}
		\lineaTabla{\alpha_c}{\num{1e9} \si{\metre^{-2}\siemens^{-1}}}{Coeficiente de creación de poros}
		\lineaTabla{V_{ep}}{0.258 \si{\volt}}{Voltaje característico}
		\lineaTabla{N_0}{\num{1.5e9} \si{\metre^{-2}}}{Densidad de poros en equilibrio}
		\lineaTabla{D}{\num{5e-14} \si{\metre^{-2}\siemens^{-1}}}{Coeficiente de difusión para poros}
		\lineaTabla{F_{max}}{\num{0.7e-3} \si{\newton\volt^{-2}}}{Máxima fuerza eléctrica}
		\lineaTabla{r_h}{\num{0.97e-9} \si{\metre}}{Constante usada para la velocidad de advección}
		\lineaTabla{r_a}{\num{0.31e-9} \si{\metre}}{Constante usada para la velocidad de advección}
		\lineaTabla{\beta}{\num{1.4e19} \si{\joule}}{Repulsión estérica}
		\lineaTabla{\gamma}{\num{1.8e11} \si{\joule\per\metre}}{Energía del perímetro de los poros}
		\lineaTabla{\sigma^\prime}{\num{2e-2} \si{\joule\metre^{-2}}}{Tensión de la interfase hidrocarburo-agua}
		\lineaTabla{\sigma_0}{\num{1e-6} \si{\joule\metre^{-2}}}{Tensión de la bicapa sin poros}
		\lineaTabla{C_m}{\num{1e-14} \si{\farad\metre^{-2}}}{Capacitancia superficial de la célula}

		\lineaTabla{F}{\num{9.648534} \si{\coulomb\per\mole}}{Constante de Faraday}
		\lineaTabla{R}{\num{8.3144621} \si{\joule\per\coulomb\per\mole}}{Constante de los gases}
		\lineaTabla{T}{310 \si{\kelvin}}{Temperatura}
		\lineaTabla{k}{\num{1.3806488e-23} \si{\joule\per\kelvin}}{Constante de Boltzmann}
		
		\hline
	\end{tabular} 
	\caption{Valores constantes usados.} %Valores obtenidos de \cite{krass}, \cite{puchiar} y \cite{marino}}
%TODO revisar valores que se hallan cambiado
	\label{table:tablita}
\end{table}

\newpage

\section{Implementación}
%
%FALTA ACOPLAMIENTO EN LA SECCIÓN MODELO MATEMÁTICO!!! (como influyen los poros en conductancia y difusión)
%FALTARÍA TAMBIÉN ACLARAR QUE LA DENSIDAD DE POROS ES MUY VARIABLE SEGÚN LA REGIÓN DE LA SUPERFICIE Y QUE LA EQ SE APLICA A CADA PORO POR SEPARADO!!

%TODO mencionar el archivo de entrada input.in!!! se menciona en otros capítulos

El problema fue dividido en tres partes, según los tres fenómenos físicos principales considerados: el potencial eléctrico en el dominio, la evolución de los poros en la membrana celular y el transporte de las especies iónicas. Las tres partes fueron implementadas y estudiadas por separado y por último fueron acopladas.\\

%TODO agregar referencias!!
%mas detalles de implementación en capítulos posteriores
El trabajo fue implementado en \texttt{C++} haciendo uso de la librería de álgebra lineal \texttt{Eigen} para resolver sistemas de ecuaciones.\\ %Los sub-problemas de potencial eléctrico y transporte de especies fueron resueltos con el método de elementos finitos. \\

Las simulaciones fueron realizadas en un equipo con procesador \texttt{Intel i3 2100} corriendo a 3.10 GHz y 8GB de memoria RAM con sistema operativo \texttt{Microsoft Windows 7}. El código es portable y fue compilado con \texttt{Microsoft Visual C++} bajo la interfaz \texttt{Microsoft Visual Studio 2013}, pero también fue probado con los compiladores \texttt{Intel C Compiler} en Windows y \texttt{GCC} en Linux.\\

Para realizar las simulaciones de los sub-problemas de potencial eléctrico y transporte de especies se utilizó el método de elementos finitos, que requiere resolver sistemas de ecuaciones lineales. Las resoluciones de los sistemas de ecuaciones se hicieron con la librería \texttt{Eigen}, usando matrices esparsas y los métodos de Cholesky y Bi-gradientes conjugados estabilizados.\\

Para acelerar los tiempos de ejecución se usó la API \texttt{OpenMP}, que consiste en directivas para correr código \texttt{C++} en paralelo (escribir mejor). 
%Más detalles de donde se usa OpenMP en los capítulos posteriores

Se usó un sistema de coordenadas cilíndricas idealizando la célula y los electrodos como sólidos de revolución. De esta manera la cantidad de nodos en la grilla empleada es considerablemente menor que si se usara un sistema de coordenadas con tres dimensiones, y los tiempos de ejecución se reducen notablemente.\\

Se generaron mallas bidimensionales con elementos cuadrilaterales usando el programa \texttt{Auto-Mesh 2D} con el método looping quad (???). Los elementos en la malla son de tamaño variable, con los elementos cercanos a la membrana de tamaño muy pequeño por ser la región de mayor interés y con cambios muy bruscos en la concentraciones y potenciales (escribir bien!). Se distinguen en la malla tres regiones: el líquido extracelular, el citoplasma (en el interior de la célula) y la membrana celular. A diferencia de otros trabajos anteriores (agregar refs) la membrana celular se modela en la malla con elementos propios del tamaño real en vez de considerarse con un ancho superior al real o directamente una condición de borde. El método de elementos finitos fue elegido en lugar de el método de diferencias finitas porque permite crear mallas con elementos de tamaños irregulares y realizar cambios en las mallas empleadas sin modificar el programa que realiza la simulación.\\

acá deberían ir un par de dibujos de la malla\\

Las mallas utilizadas tienen entre xxx y xxx elementos y representan un dominio de ??? \si{\micro\metre} de alto y ??? \si{\micro\metre} de ancho con células de ?? \si{\micro\metre} de radio y membranas de 5 \si{\nano\metre} de ancho. Para modelar la célula se dividió el ángulo $\theta$ en 192 partes con dos elementos por cada ángulo discreto (explicar mejor???).\\

También se usó \texttt{Python} como lenguaje secundario, para ayudar en la generación de mallas, la interpretación de los datos de salida obtenidos en las simulaciones y la generación de gráficos, haciendo uso de la librería \texttt{mathplotlib}.

%TODO podrían ir más detalles de entrada (input.in), salida, como usar el programa, etc o podria ir un apéndice de uso

\subsection{Método de Elementos Finitos}
%acá solo explicación a grandes rasgos??
%debería ir abajo de la section implementación?
%TODO expandir bastante

Esto está copiado del preinforme. Habría que expandir bastante. También poner detalles del trabajo.\\

El método de elementos finitos (FEM) sirve para resolver ecuaciones diferenciales de manera aproximada, discretizando el dominio en zonas pequeñas y disjuntas llamadas elementos, y resolviendo un sistema de ecuaciones lineales que obtiene la solución de las ecuaciones diferenciales en un conjunto de puntos del dominio. La aplicación del método de elementos finitos consiste en: [AG CITA!] %\cite{fem} 

\begin{itemize}
	\item Discretizar el dominio continuo en una malla formada por elementos unidos por nodos. Cada uno de estos elementos debe ser pequeño y tener una forma simple (por ejemplo triángulos o cuadriláteros). El conjunto de elementos debe ser disjunto y ocupar todo el dominio; es decir, cada punto del dominio debe estar ocupado por uno y sólo un elemento. Los vértices de los elementos se llaman nodos, y suelen ser un punto en común entre dos o más elementos. Cuántos más pequeños sean los elementos, mayor será la precisión de la solución al aplicar el método, pero se necesitarán más elementos para cubrir el dominio, y por lo tanto un mayor poder de cómputo. 
	
	\item Desarrollar para cada elemento un sistema de ecuaciones lineales que relacione los valores en los nodos. Esto se hace generalmente aplicando el método de residuos ponderados a cada uno de los elementos. El sistema resultante suele tener tantas incógnitas y ecuaciones como nodos por elemento.
%	explayarse más! funciones de forma? 
		
	\item Ensamblar todos los sistemas de ecuaciones elementales en un sistema grande, con tantas ecuaciones e incógnitas como nodos en la malla del dominio. 
	
	\item Agregar las condiciones de borde al sistema. En algunos casos se realiza este paso al generar las ecuaciones elementales, es decir antes de ensamblar el sistema.
	
	\item Resolver el sistema ensamblado con algún método de resolución de ecuaciones lineales. Dado que la matriz ensamblada es muy poco densa (muy pocos elementos distintos de cero), se suele representar con estructuras especiales para matrices dispersas. En algunos problemas la matriz generada es simétrica definida positiva, lo que permite usar métodos como descomposición de Cholesky o gradientes conjugados.
\end{itemize}

%TODO \subsection{Método de Diferencias Finitas}

\subsection{Descomposición de Cholesky}

\subsection{Descomposición BiCGSTAB}
