%\begin{center}
%\large \bf \runtitulo
%\end{center}
%\vspace{1cm}
\chapter*{\runtitulo}

\noindent 
%La electroporación consiste en la aplicación de pulsos eléctricos de corta duración y alto potencial a una célula con el objetivo de crear poros en la membrana, logrando así una permeabilización que permita el ingreso de drogas o iones a su interior. En esta tesis se presenta un modelo numérico que da cuenta de la respuesta eléctrica de la membrana a la aplicación de pulsos eléctricos, la dinámica de los poros sobre la misma y el trasporte de cuatro especies iónicas diferentes. Para resolver el conjunto de ecuaciones diferenciales acopladas involucradas se usó el método de elementos finitos en dos dimensiones con coordenadas cilíndricas, asumiendo la célula como un sólido de revolución. Las ecuaciones relacionadas a la evolución de la población de poros se resuelven con un método de Euler integrado a un solo paso. Programación distribuida basada en OpenMP se utilizó para aprovechar al máximo los procesadores multithreading actuales. Los resultados se contrastan con experimentos obtenidos de la literatura así como contra los provistos por esquemas numéricos similares.

La electroporación consiste en la aplicación de pulsos eléctricos de alta intensidad y corta duración con el objetivo de crear poros en la membrana celular, logrando así un aumento de la permeabilización que permite el ingreso de drogas o iones a su interior. La utilización de la electroporación en combinación con drogas antitumorales ha demostrado tener una significativa mayor eficacia que la terapia quimioterapéutica convencional, de allí la relevancia de estudios básicos de la interacción campos eléctricos-célula. En esta tesis se presenta un nuevo modelo numérico que describe la respuesta eléctrica de la célula, en particular la membrana celular y el transporte iónico a través de la misma, a la aplicación de pulsos eléctricos. Se asume una célula esférica sometida a pulsos eléctricos por medio de dos electrodos, constituida por cuatro especies iónicas: el ión hidrógeno (\h), el hidróxido (\oh), el catión sodio (\na) y el cloruro (\cl). Para resolver las ecuaciones diferenciales que describen el potencial electrostático y el transporte iónico se usó el método de los elementos finitos en dos dimensiones espaciales en coordenadas cilíndricas. Las ecuaciones diferenciales que describen la evolución de la población de poros se resuelven por diferencias finitas utilizando el método de Euler. Se utilizó programación distribuida basada en OpenMP para aprovechar al máximo los procesadores multithreading actuales. Los resultados se contrastan con experimentos obtenidos de la literatura demostrando una excelente correlación.

\bigskip

\noindent\textbf{Palabras claves:} membrana celular, electroporación, transporte, elementos finitos