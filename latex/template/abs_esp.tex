%\begin{center}
%\large \bf \runtitulo
%\end{center}
%\vspace{1cm}
\chapter*{\runtitulo}

\noindent %completa para después, pero primero explicar brevemente motivos que nos guían, etc
%explicar electroporación

La electroporación reversible es un método consistente en la aplicación de pulsos eléctricos de alta intensidad a una célula con el objetivo de permeabilizar su membrana creando poros, y así permitir el ingreso de drogas o moléculas de ADN a su interior. Esto permite tratar tumores con menores cantidades de drogas, reduciendo los efectos secundarios.\\

%TODO BIBLIO LOW algo de electroporation

%tipos de tratamientos ?

%lo que se hace en este trabajo
En este trabajo se simula una célula esférica a la que se le aplica un pulso eléctrico de 20\si{\milli\second} de duración a través de dos electrodos, y se estudia el ingreso al interior de la célula de 4 especies iónicas: el ión hidrógeno (\h), el hidróxido (\oh), el catión sodio (\na) y el cloruro (\cl). Para eso se tiene en cuenta el campo eléctrico producido por los electrodos, la generación y evolución de poros en la membrana celular producto de la diferencia de potencial entre el interior y exterior de la célula , y la migración de las especies mencionadas, producto de la diferencia de potencial.\\

%TODO mencionar que la densidad se calcula en distintas regiones

Las simulaciones se realizaron con el método de elementos finitos sobre mallas bidimensionales que representan el dominio sobre un sistema de coordenadas cilíndricas usando elementos cuadrilaterales.

%Se utiliza el método de elementos finitos para simular tres procesos? físicos diferentes: potencial eléctrico, generación de poros y transporte de especies, que se acoplan ???

%El trabajo está basado en otros ??? que realizan algunas de las mismas ecuaciones de manera separada

\bigskip

\noindent\textbf{Palabras claves:} Guerra, Rebelión, Wookie, Jedi, Fuerza, Imperio (no menos de 5).