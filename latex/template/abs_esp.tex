%\begin{center}
%\large \bf \runtitulo
%\end{center}
%\vspace{1cm}
\chapter*{\runtitulo}

\noindent 
La electroporación consiste en la aplicación de pulsos eléctricos de corta duración y alto potencial con el objetivo de crear poros en la membrana celular, logrando así una permeabilización que permita el ingreso de drogas o iones a su interior. En esta tesis se presenta un modelo numérico que da cuenta de la respuesta eléctrica de la membrana a la aplicación de pulsos eléctricos, la dinámica de los poros sobre la misma y el trasporte de cuatro especies iónicas diferentes. Para ello se usó el método de elementos finitos en dos dimensiones con coordenadas cilíndricas, asumiendo la célula como un sólido de revolución.
Los resultados se contrastan con experimentos obtenidos de la literatura así como contra los provistos por  esquemas numéricos similares.

\bigskip

\noindent\textbf{Palabras claves:} Membrana celular, Electroporación, transporte, elementos finitos.