%\begin{center}
%\large \bf \runtitulo
%\end{center}
%\vspace{1cm}
\chapter*{\runtitulo}

\noindent 

La electroporación reversible es un método consistente en la aplicación de pulsos eléctricos de alta intensidad a una célula con el objetivo de permeabilizar su membrana creando poros, y así permitir el ingreso de drogas o moléculas de ADN a su interior. Esto permite tratar tumores con menores cantidades de drogas, reduciendo los efectos secundarios. En este trabajo se simula una célula esférica a la que se le aplica un pulso eléctrico de 20\si{\milli\second} de duración a través de dos electrodos, y se estudia el ingreso al interior de la célula de 4 especies iónicas: el ión hidrógeno (\h), el hidróxido (\oh), el catión sodio (\na) y el cloruro (\cl). Para eso se tiene en cuenta el campo eléctrico producido por los electrodos, la generación y evolución de poros en la membrana celular producto de la diferencia de potencial entre el interior y exterior de la célula , y la migración de las especies mencionadas, producto de la diferencia de potencial.
Las simulaciones se realizaron con el método de elementos finitos sobre mallas bidimensionales que representan el dominio sobre un sistema de coordenadas cilíndricas usando elementos cuadrilaterales.

\bigskip

\noindent\textbf{Palabras claves:} Guerra, Rebelión, Wookie, Jedi, Fuerza, Imperio (no menos de 5).