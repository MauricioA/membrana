%\begin{center}
%\large \bf \runtitulo
%\end{center}
%\vspace{1cm}
\chapter*{\runtitulo}

\noindent 
La electroporación consiste en la aplicación de pulsos eléctricos de corta duración y alto potencial a una célula con el objetivo de crear poros en la membrana, logrando así una permeabilización que permita el ingreso de drogas o iones a su interior. En esta tesis se presenta un modelo numérico que da cuenta de la respuesta eléctrica de la membrana a la aplicación de pulsos eléctricos, la dinámica de los poros sobre la misma y el trasporte de cuatro especies iónicas diferentes. Para resolver el conjunto de ecuaciones diferenciales acopladas involucradas se usó el método de elementos finitos en dos dimensiones con coordenadas cilíndricas, asumiendo la célula como un sólido de revolución. Las ecuaciones relacionadas a la evolución de la población de poros se resuelven con un método de Euler integrado a un solo paso. Programación distribuida basada en OpenMP se utilizó para aprovechar al máximo los procesadores multithreading actuales. Los resultados se contrastan con experimentos obtenidos de la literatura así como contra los provistos por esquemas numéricos similares.

\bigskip

\noindent\textbf{Palabras claves:} membrana celular, electroporación, transporte, elementos finitos