\chapter{Introducción}
% CAP 1 introducción. problemática e introducción del modelo

%TODO sacar todos los newline innecesarios (ver otras tesis)

El proceso de electroporación celular tiene como objetivo permeabilizar la membrana de una célula, mediante la aplicación de un campo electromagnético externo, para lograr el transporte a través de la misma de drogas o agentes terapéuticos cargados por difusión y convección. Este proceso es utilizado en diferentes tratamientos electroquímicos de tumores \cite{c1, c2} como por ejemplo, en la electroquimioterapia (ECT) donde se utilizan drogas quimioterápicas clásicas o en el caso de la Electroterapia Génica (GET), en donde se utilizan moléculas de ADN y ARN \cite{c4}. El proceso se optimiza cuando ese campo es pulsado, dependiendo del voltaje aplicado, la duración de los pulsos y la frecuencia de los mismos \cite{c3, c4}.

La membrana está compuesta por una bicapa lipídica con su interior hidrofóbico, que actúa como una barrera altamente impermeable a la mayoría de moléculas polares, impidiendo que la mayor parte del contenido hidrosoluble de la célula salga de ella. 

A tiempo infinito  cualquier molécula difundirá a través de una bicapa lipídica libre de proteínas, a favor de su gradiente de concentración. Sin embargo la velocidad a la que una molécula difunde a través de una bicapa lipídica varía enormemente, dependiendo en gran parte del tamaño de la molécula y de su solubilidad relativa al aceite (es decir, cuanto más hidrofóbica o no polar), tanto más rápidamente difundirá a través de una bicapa \cite[p.~470-471]{c12}. Una de las funciones mas importantes de la membrana es controlar la comunicación entre el medio intracelular y el exterior a través del transporte. Dentro de la célula tienen lugar dos tipos de transporte que se llevan a cabo a través de la membrana: el transporte pasivo y el activo. 

El transporte pasivo consiste en un proceso de difusión de sustancias a través de la membrana dado por la diferencia de concentración de las mismas. Estos procesos son naturales y no requieren de energía externa. El transporte activo, en cambio, es un proceso que necesita de energía para transportar las moléculas de uno a otro lado de la membrana a través de una permeabilización natural o artificial de la misma.

La electroporación de la membrana se inicia con la aplicación de un campo eléctrico que sobre la célula genera el llamado potencial transmembranal (PTM), una diferencia de voltaje inducida sobre la membrana celular que aísla a la célula del medio exterior \cite{c5} debido a que la conductividad eléctrica de la membrana es seis (6) órdenes de magnitud más pequeña que la de los medios intra y extra celular. Este potencial inducido tiene estrecha relación con la formación de poros acuosos que conducen a través de la membrana que poseen una dinámica relacionada con el PTM \cite{c8}. Sin potencial aplicado dichos poros poseen un radio relativamente pequeño, (del orden de medio nanómetro) que sólo permiten el paso de sustancia específicas de un medio al otro producto de reacciones electroquímicas en su proximidad. La mayor o menor facilidad de las moléculas para atravesar la membrana celular dependen de la carga eléctrica y la masa molar. Moléculas pequeñas o con carga eléctrica neutra pasan la membrana más fácilmente que elementos cargados eléctricamente y moléculas grandes. Además, la membrana es selectiva, lo que significa que permite la entrada de unas moléculas y restringe la de otras.

Las moléculas pequeñas no polares se disuelven fácilmente en las bicapas lipídicas y por lo tanto difunden con rapidez a través de ellas. Las moléculas polares sin carga si su tamaño es suficientemente reducido también difunden rápidamente a través de una bicapa. Ejemplos de estas sustancias no polares son los solventes orgánicos, que presentan una polaridad alta o baja. Por ejemplo: el metanol, la acetona, el etanol, la urea, etc.

La permeabilidad depende de los siguientes factores: i) Solubilidad en los lípidos: Las sustancias que se disuelven en los lípidos (moléculas hidrófobas, no polares) penetran con facilidad en la membrana dado que está compuesta en su mayor parte por fosfolípidos. ii) Tamaño: la más grande parte de las moléculas de gran tamaño no pasan a través de la membrana. Solo un pequeño número de moléculas polares de pequeño tamaño pueden atravesar la capa de fosfolípidos. iii) Carga: Las moléculas cargadas y los iones no pueden pasar, en condiciones normales, a través de la membrana. Sin embargo, algunas sustancias cargadas pueden pasar por los canales proteicos o con la ayuda de una proteína transportadora.

Sin embargo cuando se aplica un campo eléctrico al medio, la población de poros de la membrana responden al PTM en forma dinámica, abriéndose a medida que este potencial aumenta, para después cerrarse en muchos casos o alcanzar un tamaño estable en otros siguiendo una compleja estadística analizada en \cite{c7-krass07}. En respuesta a esta apertura se modifica el coeficiente de conductividad eléctrica y el de difusión de la membrana facilitando el transporte a través de la misma \cite{c6-fodava}. Básicamente en este caso por mecanismos guiados por la difusión y la movilidad de las especies iónicas. Estos fenómenos también se relacionan con la tension elástica sobre la membrana. Un campo eléctrico aplicado sobre la misma genera mediante el tensor de Maxwell una tension local que produce una deformación en la célula, que genera que la misma tome una forma oblada o prolada, según sean los campos aplicados \cite{c13, c14}. Una vez abiertos los poros y deformada la membrana en la zona de los polos de la célula, se produce una condición adecuada para que las especies iónicas de concentración distinta en cada medio (intra y extra celular) comiencen a ingresar dentro de la célula por diferencia de concentración. 

%TODO las referencias 8-12 estan mal!!
Es necesario destacar que el modelo propuesto en esta tesis y que sigue el trabajo de diferentes autores \cite{c8, c9-fem-electro, c10, c11, c12} es un mecanismo aun no establecido con firmeza y sobre el que persisten algunos puntos que aun se deben analizar y continuar estudiando. Es por ese motivo que investigaciones de este tipo resultan valiosas ya que permiten confirmar predicciones y poner en duda suposiciones analíticas que no son tenidas en cuenta por las teóricas y deben ser estudiadas en detalle. 

%TODO revisar las referencias! (9 y 10 dos veces, además 9 no va?)
Este conjunto de simulaciones deben encararse mediante una compleja batería de modelos acoplados unos con otros y mutuamente dependientes. En esta tesis se propone por primera vez un mecanismo de funcionamiento en conjunto de todos estos fenómenos. Cada uno de estos modelos debe ser abordado por una técnica numérica adecuada. En primer lugar nos proponemos simular la distribución de potencial y campo eléctrico sobre el dominio conformado por el líquido intracelular, el liquido extracelular y la membrana mediante el método de los elementos finitos \cite{c9-fem-electro, c10}, discretizando explícitamente la membrana celular \cite{c8, c9-fem-electro}. Es necesario destacar que la membrana celular posee un espesor aproximado de entre 5 y 10 nanómetros representando un desafío numérico novedoso al intentar discretizar la misma mediante elementos finitos. En la literatura la membrana es tratada mediante una condición de contorno que separa los medios extra e intra celular \cite{c12}.

La modelización de la dinámica de creación y evolución de la población de poros sobre la membrana y el tamaño de los mismos es resuelta mediante una serie de ecuaciones diferenciales ordinarias, que se evolucionan en el tiempo mediante un algoritmo de un paso usual \cite{c15}. Con la información provista por ambos modelos, calculamos la nueva conductividad eléctrica y el coeficiente de difusión de la membrana permeabilizada, así como la distribución de tensiones sobre la misma. Tensión que retroalimenta el modelo de creación de poros y el de conductividad de membrana\cite{c13}.

Con estos resultados se resuelve el problema del transporte en todo el dominio proponiendo que a través de la membrana el mismo ocurre por el área de poros abiertos. Analizaremos la movilidad de cuatro especies iónicas presentes en el medio extra e intracelular: hidrógeno (\h), hidróxido (\oh), sodio (\na) y cloruro (\cl) resolviendo las ecuaciones de Nerts-Plank para cada especie sobre todo el dominio. Para este cálculo volveremos a utilizar el método de elementos finitos sobre el mismo mallado utilizado para recalcular la distribución de potencial eléctrico.

%TODO corregir oración abajo!! vvv
Son numerosos los parámetros relevantes a tener en cuenta cuando se realiza una simulación tan compleja. En primer lugar debemos analizar los parámetros geométricos. Es necesario explorar el adecuado dominio de resolución, el tamaño de la célula puede variar en un rango que va de los 5 micrones a los 50 micrones de diámetro. El dominio general puede abarcar un espacio circundante amplio o restringir el estudio a unos pocos micrones fuera de la membrana. Es crucial el ancho de membrana utilizado. Como ya dijimos el espesor de la membrana vive entre los 5 y los 20 nanómetros dependiendo de la célula. Esta expliracionn geometria es fundamental y a ella dedicamos una buena parte del tiempo involucrado en este trabajo. Algunos de los resultados se presentaran en el capitulo 3 y 4 del mismo. Otro parámetro fundamental dado la diferencia de escalas involucradas es el tiempo. El paso temporal de cada uno de los modelos es diferente, ya que el potencial aplicado genera un potencial sobre el dominio que varia poco en función del tiempo pero el tiempo de creación y destrucción de poros es muy pequeño, sobre todo en los procesos iniciales de desarrollo del PTM. Por último el transporte posee un tiempo característico intermedio entre ambos problemas mencionados. Determinar estas escalas y ajustarlas a modelo también llevó una buena parte de la tareas preliminares a la obtención de resultados.
%TODO es alreves, el transporte varía mas lento que la tensión

Para optimizar estos parámetros se realizó un análisis paramétrico de cada uno de los modelos por separado, algunos de los cuales, los mas relevantes, son presentados en cada una de las secciones dedicadas a los mismos. 
Un punto a destacar es el relativo al tratamiento de las mallas de elementos finitos utilizadas. Hemos seleccionado un mallador externo \cite{c11} que se adapta adecuadamente a nuestro problema pero que hemos tenido que testear y comprar con otros similares. Algunos resultados se este análisis se presentan en el capítulo 2 dedicado a los modelos numéricos utilizados.

Los resultados obtenidos del modelo general son comparados con datos experimentales existentes en la literatura. Cada uno de los modelos por separado se comparan con experimentos o resultados numéricos provistos por otros códigos. En general hemos obtenido muy buenos acuerdos con los experimentos, como se mostrará en el capitulo \ref{chap:acoplado} de esta tesis \cite{c6-fodava, c7-krass07, c8}.

 Problemas complejos proveen una gran cantidad de resultados que es necesario manipular adecuadamente para poder manipular la información que dichos datos proveen además de entender el modelo, este funcionando bien o mal. Para ello hemos perfeccionado el uso de un software de visualización abierto \cite{c16}. Se mencionarán mas detalles del mismo en la capitulo \ref{chap:modelo}.
