%\begin{center}
%\large \bf \runtitle
%\end{center}
%\vspace{1cm}
\chapter*{\runtitle}

\noindent 

%Electroporation involves the application of electrical pulses of short duration and high potential to a cell in order to create pores in the membrane, thus achieving a permeabilization that allows the entry of drugs or ions. This thesis presents a numerical model that takes in consideration the electrical response of the membrane to the application of electrical pulses, the pore dynamics over it and the transport of four different ionic species. To solve the set of coupled differential equations involved, the finite element method was used in two-dimensional cylindrical coordinates, assuming the cell as a solid of revolution. The equations related to the evolution of the population of pores are solved by a Euler method integrated into a single step. Distributed programming based on OpenMP was used to maximize current multithreading processors. The results are compared with experiments obtained from the literature as well as against those provided by similar numerical schemes.


%La electroporación consiste en la aplicación de pulsos eléctricos de alta intensidad y corta duración con el objetivo de crear poros en la membrana celular, logrando así un aumento de la permeabilización que permite el ingreso de drogas o iones a su interior. 
Electroporation involves the application of electric pulses of high intensity and short duration in order to create pores in the cell membrane, thus achieving increased permeabilization that allows the entry of drugs or ions.
%La utilización de la electroporación en combinación con drogas antitumorales ha demostrado tener una significativa mayor eficacia que la terapia quimioterapéutica convencional, de allí la relevancia de estudios básicos de la interacción campos eléctricos-célula. 
The use of electroporation in combination with antitumor drugs has been shown to have a significantly higher efficiency than conventional chemotherapeutic therapy, hence the relevance of basic studies of the electric field-cell interaction.
%En esta tesis se presenta un nuevo modelo numérico que describe la respuesta eléctrica de la célula, en particular la membrana celular y el transporte iónico a través de la misma, a la aplicación de pulsos eléctricos. 
In this thesis a new numerical model is presented, describing the electrical response of the cell, particularly the cell membrane and ion transport through it, to the application of electrical pulses.
%Se asume una célula esférica sometida a pulsos eléctricos por medio de dos electrodos, constituida por cuatro especies iónicas: el ión hidrógeno (\h), el hidróxido (\oh), el catión sodio (\na) y el cloruro (\cl). 
A spherical cell is assumed, subjected to electrical pulses by means of two electrodes and consisting of four ionic species: hydrogen ion (\h), hydroxide (\oh), sodium cation (\na) and chloride (\cl).
%Para resolver las ecuaciones diferenciales que describen el potencial electrostático y el transporte iónico se usó el método de los elementos finitos en dos dimensiones espaciales en coordenadas cilíndricas. 
To solve the differential equations describing the electrostatic potential and ion transport, the finite element method in two spatial dimensions with cylindrical coordinates is used.
%Las ecuaciones diferenciales que describen la evolución de la población de poros se resuelven por diferencias finitas utilizando el método de Euler. 
The differential equations describing the evolution of the population of pores are solved by finite differences using Euler's method.
%Se utilizó programación distribuida basada en OpenMP para aprovechar al máximo los procesadores multithreading actuales. 
Distributed programming based on OpenMP was used to maximize the usage of current multithreading processors.
%Los resultados se contrastan con experimentos obtenidos de la literatura demostrando una excelente correlación.
The results are compared with experiments obtained from the literature demonstrating excellent correlation.

\bigskip

\noindent\textbf{Keywords:} cell membrane, electroporation, transport, finite elements