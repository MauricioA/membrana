%\begin{center}
%\large \bf \runtitle
%\end{center}
%\vspace{1cm}
\chapter*{\runtitle}

\noindent 

Cell electroporation involves the application of electric pulses of short duration and high potential to a cell in order to create pores in its membrane, thus achieving a permeabilization that allows the entry of drugs or ions. Simulations of individual cells to which pulses are applied through electrodes were made. The simulations take into account the following physical phenomena: the electrical potential in the domain, the creation of pores in the membrane and the evolution of their radii and transport of four different ionic species. For this the finite element method is used in two-dimensional cylindrical coordinates, assuming the cell as a solid of revolution.

\bigskip

\noindent\textbf{Keywords:} Cell membrana, electroporation, transport, finite elements.