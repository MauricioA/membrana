\documentclass[a4paper,10pt]{article}
\usepackage[utf8x]{inputenc}
\usepackage{multirow}
\usepackage{indentfirst}
\usepackage{graphicx}

%opening
\title{LRTA* sobre Sokoban}
\author{Mauricio Alfonso}

\begin{document}

\maketitle

\begin{abstract}

\end{abstract}

% \newpage

\section{Teoría}
%donde van todoas las ecuaciones y teoria

\subsection{Potencial eléctrico}
%TODO ver tamanio de fuente math

ohm?\\

\begin{equation}
	\sigma_{elem} \nabla^{2} \phi = 0 
\end{equation}

\subsection{transporte (plank?)}

\begin{equation}
	\frac{\sigma C_i}{\sigma t} = -\nabla \cdot j_i
\end{equation}

\begin{equation}
	j_i = -\mu_i C_i \nabla \phi - D_i \nabla C_i + C_i
\end{equation}

de donde sale esto??

\subsection{generación de poros}
\begin{equation}
	\frac{dN}{dt} = \alpha e^{(V_m/V_{ep})^2} \left( 1 - \frac{N}{N_0 e^{q \left(V_m/V_{ep} \right) ^2}} \right)
\end{equation}



\section{Método Numérico / Implementación}
%en donde se explica un poco elmenetos finitos y como mallamos, y como resolvemos.

Se resolvieron los sistemas eléctrico y de transporte usando el método de elementos finitos y el de generación de poros usando diferencias finitas.

\subsection{Mallado}
Para reducir la cantidad de elementos se modeló la célula como un sólido de revolución con coordenadas cilíndricas. Se generaron mallas con elementos cuadrilaterales usando el programa AutoMesh-2D, con elementos de menor tamaño en las zonas de cercanas a la membrana celular por ser de mayor interés.
TODO falta biblio

\subsection{Método de Elementos Finitos}
breve explicación de elementos finitos (falta leer?)

\subsection{Potencial eléctrico}

\subsection{Transporte de especies}

\subsection{Diferencias finitas}

\subsection{Generación de poros}

%ordenar mejor todo


\section{Resultados}

\section{Conclusiones}


\end{document}
